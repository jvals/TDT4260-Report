\section{Background} % What does the reader need to know?

% Knowledge that is necessary to understand the proposed solution
% Who is the intened audience and what do they need to know for the
% work to be understood.

% Moore's law states that the number of transistors on a CPU will double
% every two years~\cite{moore}. Ideally, this would give us twice the
% computation power on a system every 24 months. However, because of
% limitations such as memory performance, this does not happen.

Prefetching can decrease the number of cache misses. By loading
relevant data into cache before it is needed, CPU stalls can be
avoided. However, knowing what to prefetch is not easy. The definition
of a good prefetch is that the prefetched data is used by the
processor before it is
replaced~\cite{srinivasan_davidson_tyson_2004}. This means that
prefetched data which goes unused is bad, because this data requires
valuable bandwidth. Additionally, bad prefetches might evict data from
cache which could have been used by the processor in the near future.

This paper explores ``Reference Prediction Tables'' and ``Global
History Buffers'', as described in~\cite{chen_baer_1995}
and~\cite{nesbit_smith_2005}.

