\section{Background} % What does the reader need to know?

% Knowledge that is necessary to understand the proposed solution
% Who is the intened audience and what do they need to know for the
% work to be understood.

% Moore's law states that the number of transistors on a CPU will double
% every two years~\cite{moore}. Ideally, this would give us twice the
% computation power on a system every 24 months. However, because of
% limitations such as memory performance, this does not happen.

Prefetching can decrease the number of cache misses. By loading
relevant data into cache before it is needed, CPU stalls can be
avoided. However, knowing what to prefetch is not easy. The definition
of a good prefetch is that the prefetched data is used by the
processor before it is
replaced~\cite{srinivasan_davidson_tyson_2004}. This means that
prefetched data which goes unused is bad, because this data requires
valuable bandwidth. Additionally, bad prefetches might evict data from
cache which could have been used by the processor in the near future.

\subsection{Sequential Prefetching}

As mentioned, the easiest prefetching scheme is to fetch the next
block in addition to the block the processor needs. This works well in
programs with linear flow, but it's often not enough to fetch a single
block. Increasing the prefetch degree, the number of blocks to
prefetch, can improve performance. However, if too many blocks are
fetched, the cache might evict useful blocks, which would lower the
performance. Prefetch distance is the amount of blocks between the
accessed address and the prefetched address. It can be increased to
improve performance if it happens to meet the timing requirements.
This is very task specific. 

\subsection{Strided Prefetching}

A stride directed prefetcher is an example of an instruction-based
prefetcher. They use different methods to predict how big the next
stride is going to be based on earlier memory accesses. They use
different data structures to store previous addresses, which causes
some overhead both in computing and storage. But if well implemented
they can give a high speedup as we show later in this paper.

\subsection{Reference Prediction Tables}

Reference Prediction Tables is an extension of a strided prefetcher,
which seeks to avoid bad prefetches. By introducing states,
prefetching can be delayed until the observed pattern is stable. If
the same stride is observed twice in a row, a prefetch should be
considered. Otherwise, no prefetching takes place for the requested
address.

\subsection{Global History Buffer}

A more general technique is to use a global history buffer, where each
entry has a pointer to the previous entry generated by the same load
address. Usually, the GHB is accompanied by a smaller index buffer,
which holds pointers to the most recent GHB entry. The GHB prefetcher
is general, because it can be combined with different algorithms. The
taxonomy introduced by~\cite{nesbit_smith_2005} uses the pair X/Y to
denote the methods used. X is the key used for accessing the index
table, and Y is the mechanism used for detecting addressing
patterns. The implementation described in this paper uses PC/DC, that
is index lookup with the program counter, and delta correlating for
detecting access patterns.
