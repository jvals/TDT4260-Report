\section{Methodology} % How was the evaluation performed

% Experimental setup
% All information needed to reproduce the results
% Measurement collection and validation
% Baseline and other state-of-the-art solutions

\subsection{The Simulator}

All simulations have been executed on the M5 simulator, running on a
virtual machine provided by IDI~\cite{idi}.  The M5 baseline
configuration has a two-level cache. L1 is split into a 32 KiB
instruction cache and a 64 KiB data cache. L2 is a combined 1 MiB
cache. All prefetching is done for the L2 cache. Each cache line has a
size of 64 B. The bus connecting cache to memory is 64 bits wide and
has a latency of 30 ns. The frequency of the bus is 400 MHz.

The M5 simulates Alpha 21264 which is a four-issue superscalar
architecture with out-of-order execution. Alpha is able to have up to
80 instructions in partial states of completion at any
time~\cite{kessler_1999}.

\subsection{SPEC CPU2000}

In order to test the performance of the implementation, a subset of the SPEC
CPU2000 benchmarks was used~\cite{spec2000}. This includes:
\begin{multicols}{2}
\begin{itemize}
\item twolf
\item bzip2\_source
\item swim
\item bzip2\_graphic
\item bzip2\_program
\item applu
\item apsi
\item art110
\item art470
\item galgel
\item wupwise
\item ammp
\end{itemize}
\end{multicols}

These benchmarks all have different memory access patterns, which is
helpful for discovering weaknesses in the solution.
