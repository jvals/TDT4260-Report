\section{Proposed solution}\todo{Think of a better heading}

% Explain how the proposed solution works
% Give enough details to be understood but no more than necessary to
% keep the message clear.
% Enough information for the solution to be implementable
% Use illustrations when possible to convey the message

\subsection{Sequential prefetcher}
To be conclusive, an implementation of the simlpest for of prefetcher
was done. Whenever a miss occures the prefetcher fecthes the object
that is next in memory. We tried a couple of different settings for
prefecth distance and prefetch degree.

\subsection{Stride Directed Prediction}
A slight impprovement from a sequential prefetcher is the
\textit{Stride Directed prefetcher, (SDP)}~\cite{sdp}.
To avoid cache pollution in programs that have non-sequential memory
accessing SDP was introduced. It utilises a simple table to
predict how big the next stride is going to be. The first time a load
instruction is encountered \textit{Program Counter (PC)} and the
accessed address is stored in a table entry as in
Table~\ref{table:sdpentry}. On the next access \textit{Last Adress}
and current address is compared. The difference, ``Stride'', is added
to the current address and to which a prefetch is issued.

The total size of an SDP entry is 24 byte, 192 bits, even though the
fields sum up to 129 bits. This is due to padding added to satisfy
alignment constraints.

\begin{table}[h]
  \centering
  \begin{tabular}{ | c | c | c |}
    \hline
    PC & Last Adress & Valid Bit \\ \hline
    64 bit & 64 bit & 1 bit \\ \hline
  \end{tabular}
  \caption{SDP entry}
  \label{table:sdpentry}
\end{table}


\subsection{Reference Prediction Tables}

RPT makes use of a table with entries corresponding to load
instructions. Every time the program counter hits a load, the current
address (tag), as well as the memory location loaded from the previous
address, is stored as an entry in this table. When a tag which already
resides in the table is found, the program checks what address is about
to be loaded, and from that, subtract the previous address. The result
of this subtraction is stored in the corresponding entry as the
``stride''. The complete format of an entry is shown in
Table~\ref{table:entry}.

\begin{table}[h]
  \centering
  \begin{tabular}{ | c | c | c | c |}
    \hline
    Tag & Previous Address & Stride & State \\ \hline
    64 bit & 64 bit & 32 bit & 32 bit \\ \hline
  \end{tabular}
  \caption{RPT Entry}
  \label{table:entry}
\end{table}

If two consecutive strides are equal, this indicates a useful access
pattern. A prefetch is issued if the block is not in cache.

To avoid bad prefetches, a state machine which indicates the current
status of the system is used. Initially, an entry has state
``init''. Every time an entry which already resides in the table is
encountered, a check is done to see if the stride is equal to the one
stored in the entry. That is, if current\_address$ -
$previous\_address = stride. If this is true (correct), a transition
to a new state occurs. Otherwise (incorrect), there is no
transition. The state machine is shown in
Figure~\ref{figure:statemachine}. If the entry's state is init,
transient or steady, a prefetch is considered.

An improvement on this design is to issue multiple prefetches at the
same time. This is done to prevent data arriving too late, or being
replaced too early. A variable \texttt{times} is introduced (shown in
Table~\ref{table:entry_la}) and initially set to 1. Each time there is
a table hit, \texttt{times} is incremented (up to a fixed lookahead
limit) if the correct-condition is true. If not, it is decremented
(down to a limit of 1). Instead of issuing a single prefetch, a batch
of prefetches with address = previous\_address$ + $stride$\times$i,
where i $= [1, 2, \ldots, \texttt{times}]$ are issued.

\begin{table}[h]
  \centering
  \begin{tabular}{ | c | c | c | c | c |}
    \hline
    Tag & Previous Address & Stride & State & Times \\ \hline
    64 bit & 64 bit & 32 bit & 32 bit & 8 bit \\ \hline
  \end{tabular}
  \caption{RPT Entry with lookahead}
  \label{table:entry_la}
\end{table}

% State machine
\begin{figure}[h!]
\begin{center}
\begin{tikzpicture}[node distance=3cm,on grid,auto]
   \node[state] (q_0)   {init};
   \node[state] (q_1) [right=of q_0] {steady};
   \node[state] (q_2) [below=of q_0] {transient};
   \node[state] (q_3) [right=of q_2] {no-pred};
    \path[->]
    (q_0) edge  [out=5, in=175] node {correct} (q_1)
          edge  node [swap, text width=2.3cm] {\hspace{0.4cm}incorrect
            (update stride)} (q_2)
    (q_1) edge [out=185, in=355, looseness=1] node {incorrect} (q_0)
          edge [loop right] node {correct} ()
    (q_2) edge  [out=355, in=185] node [swap, text width=2.3cm] {\hspace{0.4cm}incorrect (update stride)} (q_3)
          edge  node [swap] {correct} (q_1)
    (q_3) edge [out=175, in=5] node [swap] {correct} (q_2)
          edge  [loop right] node [text width=2.3cm] {\hspace{0.4cm}incorrect (update stride)} ();
\end{tikzpicture}
\caption{State machine as proposed in~\cite{chen_baer_1995}}
\label{figure:statemachine}
\end{center}
\end{figure}

\subsection{Global History Buffer with PC/DC}

A global history buffer (GHB) is a FIFO table where each entry is an
element of a linked list. This means that an entry contains a pointer
to the previous entry generated by the same address. Every time an
address is requested, it is pushed into the GHB.

A smaller table, the index table, is used to keep pointers into the
GHB. The index table is accessed the same way as in other table-based
prefetchers. In this case, the address of the load instruction acts as
the key. Each index table entry has a PC field and a ghb pointer
field.

After accessing the index table, if the PC field does not match the PC
of the load instruction, the field is updated. However, if the PC
field does match, the current ghb entry's back pointer is set to the
address of the ghb pointer. Finally, the ghb pointer is set to the
head of the GHB buffer.

A delta table is built by traversing the back pointers and storing the
difference (delta) between consecutive miss addresses. The delta table
is traversed to find a delta pair equal to the first pair of deltas,
this is called delta correlating. If a pattern is found, a prefetch is
issued for the address equal to delta$+$current miss address, where
delta is every delta between the matching pairs.
